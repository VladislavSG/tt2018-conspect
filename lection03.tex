\section{Лекция 3}

\subsection{Y-комбинатор}

\begin{definition}($Y$-комбинатор)
	\[
	Y = \lambda f . (\lambda x . f (x x)) (\lambda x . f (x x))
	\]
\end{definition}

Очевидно, $Y$-комбинатор является комбинатором.

\begin{theorem}
	$Y f =_{\beta} f (Y f)$
	
	\begin{proof}
		$\beta$-редуцируем выражение $Y f$
		\begin{align*}
			 =_{\beta} \textcolor{magenta}{(\lambda f . (\lambda x . f (x x)) (\lambda x . f (x x)))}\textcolor{blue}{f} \\ =_{\beta} \textcolor{magenta}{(\lambda x . f (x x))} \textcolor{blue}{(\lambda x . f (x x))} \\ =_{\beta} f ((\lambda x . f (x x))(\lambda x . f (x x))) \\ =_{\beta} f(Y f)
		\end{align*}
		Так как при второй редукции мы получили, что $Y f =_{\beta} (\lambda x . f (x x))(\lambda x . f (x x))$
	\end{proof}	
\end{theorem}

Следствием этого утверждения является теорема о неподвижной точке для бестипового $\lambda$-исчисления

\begin{theorem}
	В $\lambda$-исчислении каждый терм $f$ имеет неподвижную точку, то есть такое $p$, что $f \; p =_{\beta} p$
	
	\begin{proof}
		Возьмём в качестве $p$ терм $Y f$. По предыдущей теореме, $f(Y f) =_{\beta} Y f$, то есть $Y f$ является неподвижной точкой для $f$. Для любого терма $f$ существует терм $Y f$, значит, у любого терма есть неподвижная точка.
	\end{proof}
\end{theorem}

\subsection{Рекурсия}

С помощью $Y$-комбинатора можно определять рекурсивные функции, например, функцию, вычисляющую факториал Чёрчевского нумерала. Для этого определим вспомогательную функцию

$fact' \equiv \lambda f. \lambda n. isZero\; n \; \overline{1} (mul \; n \; f((-1) n))$

Тогда $fact \equiv Y fact'$

Заметим, что $fact \; \overline{n} =_{\beta} fact' \; (Y \; fact') \; \overline{n} =_{\beta}fact' \; fact \; \overline{n} $, то есть в тело функции $fact'$ вместо функции $f$ будет подставлена $fact$ (заметим, что это значит, что именно функция $fact$ будет применена к $\overline{n - 1}$, то есть это соответствует нашим представлениям о рекурсии).

Для понимания того, как это работает, посчитаем $fact \; \overline{2}$

\begin{align*}
	fact \; \overline{2} \\ =_{\beta} Y \; fact' \; \overline{2}\\ =_{\beta} fact' (Y \; fact') \overline{2} \\=_{\beta}\textcolor{magenta}{(\lambda f. \lambda n. isZero\; n \; \overline{1} (mul \; n \; f((-1) n)))} \textcolor{blue}{(Y \; fact') \overline{2}} \\
	=_{\beta}isZero\; \overline{2} \; \overline{1} (mul \; \overline{2} \; ((Y \; fact')((-1) \overline{2}))) \\ =_{\beta} mul \; \overline{2} \; ((Y \; fact')((-1) \overline{2})) \\ =_{\beta} mul \; \overline{2} \; (Y \; fact' \; \overline{1}) \\ =_{\beta} mul \; \overline{2} \; (fact' \;(Y \; fact' \; \overline{1}))
\end{align*}

Раскрывая $fact' \;(Y \; fact' \; \overline{1})$ так же, как мы раскрывали  $fact' \;(Y \; fact' \; \overline{2})$, получаем

\begin{align*}
	=_{\beta} mul \; \overline{2} \; (mul \; \overline{1} \; (Y \; fact' \; \overline{0}))
\end{align*}

Посчитаем $(Y \; fact' \; \overline{0})$

\begin{align*}
	(Y \; fact' \; \overline{0}) \\ =_{\beta} fact' \; (Y \; fact') \; \overline{0} \\ =_{\beta} \textcolor{magenta}{(\lambda f. \lambda n. isZero\; n \; \overline{1} (mul \; n \; f((-1) n)))} \; \textcolor{blue}{(Y \; fact') \; \overline{0}} \\ =_{\beta}  isZero\; \overline{0} \; \overline{1} (mul \; \overline{0} \; ((Y \; fact'))((-1) \overline{0})) =_{\beta} \overline{1}
\end{align*}

Таким образом,

\begin{align*}
	fact \; \overline{2} \\ =_{\beta} mul \; \overline{2} \; (mul \; \overline{1} \; (Y \; fact' \; \overline{0})) \\=_{\beta} mul \; \overline{2} \; (mul \; \overline{1} \; \overline{1}) =_{\beta} mul \; \overline{2} \; \overline{1} =_{\beta} \overline{2}
\end{align*}

\subsection{Парадокс Карри}

Попробуем построить логику на основе $\lambda$-исчисления. Введём логический символ $\rightarrow$. 

Будем требовать от этого исчисления наличия следующих схем аксиом:

\begin{enumerate}
	\item $\vdash A \rightarrow A$
	\item $\vdash (A \rightarrow (A \rightarrow B)) \rightarrow (A \rightarrow B)$
	\item $\vdash A =_{\beta} B$, тогда $A \rightarrow B$
\end{enumerate}

А также правила вывода MP:

$$\infer{\vdash B}{\vdash A \rightarrow B && \vdash A}$$

Не вводя дополнительные правила вывода и схемы аксиом, покажем, что данная логика является противоречивой. Для чего введём следующие условные обозначения:

$F_{\alpha} \equiv \lambda x. (x \; x) \rightarrow \alpha$

$\Phi_{\alpha} \equiv F_{\alpha}  \; F_{\alpha}  \equiv (\lambda x. (x \; x) \rightarrow \alpha) \; (\lambda x. (x \; x) \rightarrow \alpha)$

Редуцируя $\Phi_{\alpha}$, получаем 

\begin{align*}
\Phi_{\alpha} \\ =_{\beta} \textcolor{magenta}{(\lambda x. (x \; x) \rightarrow \alpha)} \; \textcolor{blue}{(\lambda x. (x \; x) \rightarrow \alpha)} \\=_{\beta} (\lambda x. (x \; x) \rightarrow \alpha) \; (\lambda x. (x \; x) \rightarrow \alpha) \rightarrow \alpha \\=_{\beta} \Phi_{\alpha} \rightarrow \alpha
\end{align*}

Теперь докажем противоречивость введённой логики. Для этого докажем, что в ней выводимо любое утверждение.

\begin{tabular}{ll}
	1) $\vdash\Phi_\alpha\rightarrow\Phi_\alpha\rightarrow\alpha$ & Так как $\Phi_{\alpha} =_{\beta} \Phi_{\alpha} \rightarrow \alpha$\\
	2) $\vdash(\Phi_\alpha\rightarrow\Phi_\alpha\rightarrow\alpha)\rightarrow(\Phi_\alpha\rightarrow\alpha)$ & Так как $\vdash (A \rightarrow (A \rightarrow B)) \rightarrow (A \rightarrow B)$\\
	3) $\vdash\Phi_\alpha\rightarrow\alpha$ & MP 1, 2\\
	4) $\vdash (\Phi_\alpha \rightarrow \alpha) \rightarrow \Phi_\alpha$ & Так как $\vdash \Phi_\alpha \rightarrow \alpha =_{\beta} \Phi_\alpha$\\
	5) $\vdash\Phi_\alpha$ & MP 3, 4\\
	6) $\vdash\alpha$ & MP 3, 5
\end{tabular}

Таким образом, введённая логика оказывается противоречивой.

\subsection{Импликационный фрагмент интуиционистского исчисления \\высказываний}

Рассмотрим подмножество ИИВ, со следующей грамматикой:

$\Phi ::= x \; | \; \Phi \rightarrow \Phi \; | \; (\Phi)$

То есть состоящее только из переменных и импликаций. 

Добавим в него одну схему аксиом

$$\Gamma, \varphi \vdash \varphi$$

И два правила вывода

\begin{enumerate}
	\item Правило введения импликации:
	\[
	\infer{\Gamma \vdash \varphi \to \psi}{\Gamma, \varphi \vdash \psi}
	\]
	\item Правило удаления импликации:
	\[
	\infer{\Gamma \vdash \psi}{\Gamma \vdash \varphi \to \psi && \Gamma \vdash \varphi}
	\]
\end{enumerate}

\begin{example}
	Докажем $\vdash \varphi \rightarrow \psi \rightarrow \varphi$
	
	\[
	\infer[(\text{Введение импликации})]
	{ \vdash \varphi \to (\psi \to \varphi) }
	{ \infer[(\text{Введение импликации})]
		{ \varphi \vdash \psi \to \varphi }
		{\varphi, \psi \vdash \varphi}
	}
	\]
\end{example}

\begin{example}
	Докажем $\alpha \rightarrow \beta \rightarrow \gamma, \; \alpha, \; \beta \vdash \gamma$
	
	\[
	\infer
	{ \alpha \rightarrow \beta \rightarrow \gamma, \; \alpha, \; \beta \vdash \gamma}{\infer
		{\alpha \rightarrow \beta \rightarrow \gamma, \; \alpha, \; \beta \vdash \beta \rightarrow \gamma }{\alpha \rightarrow \beta \rightarrow \gamma, \; \alpha, \; \beta \vdash \alpha \rightarrow \beta \rightarrow \gamma && \alpha \rightarrow \beta \rightarrow \gamma, \; \alpha, \; \beta \vdash \alpha} && \alpha \rightarrow \beta \rightarrow \gamma, \alpha, \; \beta \vdash \beta}
	\]
	
\end{example}

\begin{note}
	В дальнейшем символом $\vdash_\rightarrow$ будем обозначать доказуемость в импликационном фрагменте.
\end{note}

\begin{theorem} Замкнутость И.Ф.ИИВ
	
	Если $\Gamma$ и $\varphi$ состоят только из импликаций, то $\Gamma \vdash \varphi$ равносильна $\Gamma \vdash_\rightarrow \varphi$.
\end{theorem}

\begin{lemma}
	\label{conj}
	Если $\Gamma \vdash \varphi$, то в любой модели Крипке из $\Vdash \Gamma$ следует $\Vdash \varphi$
\end{lemma}
\begin{proof}\ \\
	Пусть $\Gamma \vdash \varphi$. 
	Тогда $\vdash \& \Gamma \rightarrow \varphi$, где $\& \Gamma$ --- конъюнкция всех утверждений в $\Gamma$.
	По корректности моделей Крипке, будет выполнено $\Vdash \& \Gamma \rightarrow \varphi$. Переписывая $\&$ и $\rightarrow$ по определению, получаем $\Vdash \Gamma \ \Rightarrow \ \Vdash \varphi$.
\end{proof}

\begin{theorem}
	\label{kripke}
	$\Gamma \vdash_\rightarrow \varphi$ т.и.т.т. в любой модели Крипке из $\Vdash \Gamma$ следует $\Vdash \varphi$
\end{theorem}
\begin{proof}\
	\begin{itemize}
		\item $(\Rightarrow)$ Очевидно по лемме \ref{conj}.
		\item $(\Leftarrow)$ Пусть в любой модели Крипке из $\Vdash \Gamma$ следует $\Vdash \varphi$. Докажем $\Gamma \vdash_\rightarrow \varphi$.
		
		Выберем подходящую модель Крипке. Напомним, что моделью Крипке называется тройка $\langle C, \succcurlyeq, \Vdash\rangle$, где $C$ -- множество миров, $\succcurlyeq$ -- отношение частичного порядка на $C$, $\Vdash$ -- отношение вынужденности переменной.
		
		\textbf{Построим модель Крипке} $\mathcal C = \langle C, \succcurlyeq, \Vdash \rangle$. 
		
		Пусть $C = \{\Delta \ |\ \Gamma \subseteq \Delta, \Delta \text{ замкнут относительно }\vdash_\rightarrow\}$. $\Delta$ замкнут относительно доказуемости, когда для любого $\varphi$ если $\Delta \vdash_\rightarrow \varphi$, то $\varphi \in \Delta$.
		
		$C_1 \succcurlyeq C_2$, если $C_1 \supseteq C_2$.
		
		$\Delta \Vdash \alpha$, если $\alpha \in \Delta$, $\alpha$ -- переменная.
		
		\begin{lemma}
			\label{mymodel}
			В модели $\mathcal C$ $\Delta \Vdash \varphi$ т.и.т.т. $\varphi \in \Delta$
		\end{lemma}
		\begin{proof}
			Индукция по структуре $\varphi$. 
			
			\textbf{База.} $\varphi \equiv \alpha$. $\Delta \Vdash \alpha \Leftrightarrow \alpha \in \Delta$ следует из определения вынужденности.
			
			\textbf{Индукционный переход.} $\varphi \equiv \psi \rightarrow \sigma$
			
			Индукционное предположение:
			$\forall \Delta \in C: \Delta \Vdash \psi \Leftrightarrow \psi \in \Delta$, $\Delta \Vdash \sigma \Leftrightarrow \sigma \in \Delta$.
			
			Докажем, что $\Delta \Vdash \psi \rightarrow \sigma \Leftrightarrow \psi \rightarrow \sigma \in \Delta$.
			\begin{itemize}
				\item $(\Rightarrow)$ Пусть $\Delta \Vdash \psi \rightarrow \sigma$.
				
				Рассмотрим мир $\Pi = (\Delta \cup \{\psi\})^*$. $\Pi \Vdash \psi \rightarrow \sigma$, т.к. $\Delta \preccurlyeq \Pi$.
				
				$\psi \in \Pi$. Тогда, по инд. пред., $\Pi \Vdash \psi$. Значит, $\Pi \Vdash \sigma$. В самом деле, из определения вынужденности импликации в $\Pi$ следует, что если $\Pi \Vdash \psi$, то $\Pi \Vdash \sigma$.
				
				По инд. пред. заключаем $\sigma \in \Pi$, т.е. $\Pi \vdash_\rightarrow \sigma$, т.к. $\Pi$ -- замкнут по доказуемости. Ясно, что $\Delta, \psi \vdash_\rightarrow \sigma$. Действительно, в гипотезах доказательства $\Pi \vdash_\rightarrow \sigma$ использовалось не все бесконечное множество $\Pi$, а лишь конечный набор утверждений из него. Каждое такое утверждение выводится из $\Delta, \psi$, потому что $\Pi$ - замыкание $\Delta \cup \{\psi\}$. 
				
				Из $\Delta, \psi \vdash_\rightarrow \sigma$ следует $\Delta \vdash_\rightarrow \psi \rightarrow \sigma$. Таким образом, $\psi \rightarrow \sigma \in \Delta$.
				
				\item $(\Leftarrow)$ Пусть $\psi \rightarrow \sigma \in \Delta$.
				
				Рассмотрим произвольный мир $\Pi: \Delta \preccurlyeq \Pi \land \Pi \Vdash \psi$. По инд. пред. $\psi \in \Pi$. $\psi \rightarrow \sigma \in \Pi$, т.к. $\Delta \subseteq \Pi$. $\Pi \vdash_\rightarrow \psi$, $\Pi \vdash_\rightarrow \psi \rightarrow \sigma$. Очевидно, $\Pi \vdash_\rightarrow \sigma$. $\sigma \in \Pi$. Тогда, по инд. пред., $\Pi \Vdash \sigma$. Таким образом, $\Pi \Vdash \psi \rightarrow \sigma$, а следовательно, $\Delta \Vdash \psi \rightarrow \sigma$.
			\end{itemize}
		\end{proof}
		$\mathcal C \Vdash \Gamma$, т.к. любой мир модели $\mathcal C$ -- надмножество $\Gamma$. В любой модели Крипке из $\Vdash \Gamma$ следует $\Vdash \varphi$. В частности, это выполнено в модели $\mathcal C$. Значит $\mathcal C \Vdash \varphi$, в том числе $\Gamma^* \Vdash \varphi$. По лемме \ref{mymodel} $\varphi \in \Gamma^*$, то есть $\Gamma^* \vdash_\rightarrow \varphi$. Следовательно, $\Gamma \vdash_\rightarrow \varphi$.
	\end{itemize}
\end{proof}

Теперь можем доказать теорему о замкнутости ИФИИВ.
\begin{proof}\ \\
	Следствие $\Gamma \vdash_\rightarrow \varphi \ \Rightarrow \ \Gamma \vdash \varphi$ очевидно.\\
	Пусть $\Gamma \vdash \varphi$. По \ref{conj} получаем, что в любой модели Крипке из $\Vdash \Gamma$ следует $\Vdash \varphi$. Отсюда, по теореме \ref{kripke}, доказывается $\Gamma \vdash_\rightarrow \varphi$.
\end{proof}

\subsection{Просто типизированное по Карри $\lambda$-исчисление}

\begin{definition}
	Тип в просто типизированном $\lambda$-исчислении по Карри --- это либо маленькая греческая буква ($\alpha, \phi, \theta, \ldots$), либо импликация ($\theta_1 \rightarrow \theta_2$)
	
	Таким образом, $\Theta ::= \theta_{i} | \Theta \rightarrow \Theta | (\Theta)$
	
	Импликация при этом считается правоассоциативной операцией.
\end{definition}

\begin{definition}
	Язык просто типизированного $\lambda$-исчисления --- это язык бестипового $\lambda$-исчисления.
\end{definition}

\begin{definition}
	Контекст $\Gamma$ --- это список выражений вида $A: \theta$, где $A$ --- $\lambda$-терм, а $\theta$ --- тип.
\end{definition}

\begin{definition}
	Просто типизированное $\lambda$-исчисление по Карри.
	
	Рассмотрим исчисление с единственной схемой аксиом:
	
	$$\Gamma, x : \theta \vdash x : \theta, \text{если } x \text{ не входит в } \Gamma$$
	
	И следующими правилами вывода
	
	\begin{enumerate}
		\item Правило типизации абстракции
		\[
		\infer[\text{если } x \text{ не входит в } \Gamma]{\Gamma \vdash (\lambda \; x. \; P) : \varphi \rightarrow \psi}{\Gamma, x : \varphi \vdash P : \psi}
		\]
		\item Правило типизации аппликации:
		\[
		\infer{\Gamma \vdash PQ : \psi}{\Gamma \vdash P : \varphi \to \psi && \Gamma \vdash Q : \varphi}
		\]
	\end{enumerate}

	Если $\lambda$-выражение типизируется с использованием этих двух правил и одной схемы аксиом, то будем говорить, что оно типизируется по Карри.
\end{definition}

\begin{example}
	Докажем $\vdash \lambda \; x. \; \lambda \; y. \; x : \alpha \rightarrow \beta \rightarrow \alpha$
	
	\[
	\infer[(\text{Правило типизации абстракции})]
	{ \vdash \lambda \; x. \; \lambda \; y. \; x : \alpha \rightarrow \beta \rightarrow \alpha }
	{ \infer[(\text{Правило типизации абстракции})]
		{x: \alpha \vdash \lambda \; y. \; x : \beta \rightarrow \alpha}
		{x: \alpha, y : \beta \vdash x : \alpha}
	}
	\]
\end{example}


\begin{example}
	Докажем $\vdash \lambda \; x. \; \lambda \; y. \; x \; y : (\alpha \rightarrow \beta) \rightarrow \alpha \rightarrow \beta$
	
	\[
	\infer
	{\vdash \lambda \; x. \; \lambda \; y. \; x \; y : (\alpha \rightarrow \beta) \rightarrow \alpha \rightarrow \beta}
	{
		\infer
		{x: \alpha \rightarrow \beta \vdash \lambda \; y. \; x \; y : \alpha \rightarrow \beta}
		{
			\infer
			{x: \alpha \rightarrow \beta, y : \alpha \vdash x \; y : \beta}
			{x: \alpha \rightarrow \beta, y : \alpha \vdash x: \alpha \rightarrow \beta && x: \alpha \rightarrow \beta, y : \alpha \vdash y: \alpha}
		}
	}
	\]
\end{example}

\subsection{Отсутствие типа у Y-комбинатора}

\begin{theorem}
	$Y$-комбинатор не типизируется в просто типизированном по Карри $\lambda$-исчислении.
\end{theorem}

\subparagraph{Неформальное доказательство}

$Y \; f =_{\beta} f \; (Y \; f)$, поэтому $Y \; f$ и $f \; (Y \; f)$ должны иметь одинаковые типы.

Пусть $Y \; f : \alpha$

Тогда $Y : \beta \rightarrow \alpha, f : \beta$

Из $f \; (Y \; f) : \alpha$ получаем $f: a \rightarrow \alpha$ (так как $Y f : \alpha$)

Тогда $\beta = \alpha \rightarrow \alpha$, из этого получаем $Y : (\alpha \rightarrow \alpha) \rightarrow \alpha$

Можно доказать, что $\lambda \; x. \; x : \alpha \rightarrow \alpha$. Тогда $Y \; \lambda \; x. \; x : \alpha$, то есть любой тип является обитаемым. Так как это невозможно, $Y$-комбинатор не может иметь типа, так как тогда он сделает нашу логику противоречивой.

\subparagraph{Формальное доказательство}

Докажем от противного. Пусть $Y$-комбинатор типизируем. Тогда в выводе его типа есть вывод типа выражения $x \; x$. Так как $x \; x$ --- абстракция, то и типизирована она может быть только по правилу абстракции. Значит, в выводе типа $Y$-комбинатора есть такой вывод:

$$\infer{\Gamma \vdash x x: \psi}{\Gamma \vdash x: \varphi \rightarrow \psi && \Gamma \vdash x: \varphi }$$

Рассмотрим типизацию $\Gamma \vdash x: \varphi \rightarrow \psi$ и $\Gamma \vdash x: \varphi$. $x$ это атомарная переменная, значит, она могла быть типизирована только по единственной схеме аксиом. 

Следовательно, $x$ типизируется следующим образом.

$$\infer{\Gamma', x: \varphi \rightarrow \psi, x: \varphi \vdash x x: \psi}{\Gamma', x: \varphi \rightarrow \psi, x: \varphi \vdash x: \varphi \rightarrow \psi && \Gamma', x: \varphi \rightarrow \psi, x: \varphi \vdash x: \varphi }$$

Следовательно, в контексте $\Gamma$ переменная $x$ встречается два раза, что невозможно по схеме аксиом.

\subsection{Изоморфизм Карри-Ховарда}

Заметим, что аксиомы и правила вывода импликационного фрагмента ИИВ и просто типизированного по Карри $\lambda$-исчисления точно соответствуют друг другу. 

$\newline$
\begin{tabular}{ | p{8cm} | p{8cm} | }
	\hline
	Просто типизированное $\lambda$-исчисление & Импликативный фрагмент ИИВ \\ \hline
	$\Gamma, x : \theta \vdash x : \theta$ & $\Gamma, \varphi \vdash \varphi$ \\
	&\\
	$\infer{\Gamma \vdash (\lambda \; x. \; P) : \varphi \rightarrow \psi}{\Gamma, x : \varphi \vdash P : \psi}$ & $\infer{\Gamma \vdash \varphi \to \psi}{\Gamma, \varphi \vdash \psi}$  \\
	&\\
	$\infer{\Gamma \vdash PQ : \psi}{\Gamma \vdash P : \varphi \to \psi && \Gamma \vdash Q : \varphi}$ & $\infer{\Gamma \vdash \psi}{\Gamma \vdash \varphi \to \psi && \Gamma \vdash \varphi}$ \\
	\hline
\end{tabular}

$\newline$
Установим соответствие и между прочими сущностями ИИВ и просто типизированного по Карри $\lambda$-исчисления.

$\newline$
\begin{tabular}{ | p{8cm} | p{8cm} | }
	\hline
	Просто типизированное $\lambda$-исчисление & Импликативный фрагмент ИИВ \\ \hline
	Тип & Высказывание \\
	Терм & Доказательство высказывания  \\
	Проверка того, что терм имеет заданный тип & Проверка доказательства на корректность \\
	Обитаемый тип & Доказуемое высказывание \\
	Проверка того, что существует терм, имеющий заданный тип & Проверка того, что заданное высказывание имеет доказательство \\
	\hline
\end{tabular}
